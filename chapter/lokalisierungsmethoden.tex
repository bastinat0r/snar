% vim:tw=78:ai:bg=light:set spell:spelllang=de:set nu
%%%%%%%%%%%%%%%%%%%%%%%%%%%%%%%%%%%%%%%%%%%%%%%%%%%%%%%%%%%%%%%%%%%%%%%%%%%%
%%% Lokalisierungsmethoden
%%%%%%%%%%%%%%%%%%%%%%%%%%%%%%%%%%%%%%%%%%%%%%%%%%%%%%%%%%%%%%%%%%%%%%%%%%%%
\chapter{Lokalisierungsmethoden}

\section{Globales Navigationssatellitensystem}
Das wohl bekannteste Verfahren eine Position zu ermitteln ist derzeit das
\dq Global Navigation Satellite System\dq. Der Kern dieses Systems sind Satelliten im Erdorbit, die
kontinuierlich Funksignale zur Erde senden. Diese werden dort von einem
entsprechenden Empfänger aufgenommen und mit Hilfe von mathematischen
Berechnungen zu einer globalen Position vereinigt.

Soll dieses Verfahren jedoch verwendet werden um eine Indoor-Lokalisierung
zu realisieren, steht man vor mehreren Problemen. Die
Funksignale der Satelliten sind durch Hindernisse nur noch recht schlecht
empfangbar, da deren Intensität je nach Material abnimmt. Um die Signale
einigermaßen gut empfangen zu können, werden spezielle Sensoren benötigt, was
auch einen höheren Geldaufwand bedeutet.

Ein weiteres Problem ist die große Ungenauigkeit von GNSS in Räumen. Auf Grund
von Reflexionen entstehen Fehler, die zu einer Ungenauigkeit in der
Lokalisierung führen. Die Anforderung von Lokalisierung in Räumen verlangt
eine recht hohe Auflösung, da im Verhältnis zu Draußen oft nur wenig Zeit
zum Reagieren bleibt und die Abmaße der Gebiete viel kleiner sind. Da ist
eine Abweichung von mehreren Metern nicht hinnehmbar. GNSS würde sich also
nur anbieten, wenn es darum geht, zu bestimmen in welchem Raum man sich
befindet. Um dabei eine schnelle Positionierung zu erreichen, wird das
\dq Assisted Global Positioning System\dq{} verwendet. Dabei verfügt der Empfänger über eine Datenverbindung
zu einer Informationsquelle in der Informationen darüber zufinden sind,
wo sich zum aktuellen Zeitpunkt welche Satelliten befinden. Er bekommt somit
eine hilfreiche Information, die er sonst über die Signale der Satelliten
selbst berechnen müsste. Um in Räumen eine Aussage darüber treffen zu können,
ob ein empfangenes Signal gültig ist, wird über mehrere Intervalle integriert.

\section{Vorgetäuschtes GNSS}
Bei vorgetäuschtem GNSS werden spezielle Transceiver aufgestellt, die
GNSS-ähnliche Signale aussenden. Die Transceiver sind dabei
zeitsynchronisiert. Dieses System kann sowohl indoor als auch outdoor
verwendet werden. Es erreicht dabei eine eine maximale Abweichung von
$\pm$20 mm. In Gebäuden eignet sich dieses System sehr gut für eine
Lokalisierung, da die Signale eine ausreichende Intensität haben um Wände
zu durchdringen. Indoor wird dabei jedoch nur eine Genauigkeit im
Dezimeterbereich erzielt.

\section{Laserverfolgung}
Bei diesem Verfahren werden die Objekte, von denen die Positionen
bestimmt werden sollen, mit Reflektoren ausgestattet und dann von einem
Laserinterferometer verfolgt. Aus den gesammelten Daten kann dann eine
Position bestimmt werden. Das Prinzip ist ähnlich dem der Landvermessung.
Die Reichweite kann bei der Laserverfolgung bis zu 70m betragen. Zusätzlich
besteht die Möglichkeit bewegte Objekte dauerhaft zu verfolgen, um eine
kontinuierliche Positionsbestimmung zu gewährleisten. Die Genauigkeit liegt
bei 0,001 Zoll.

\section{Funksignale}
Um Funksignale für die Lokalisierung zu verwenden, gibt es mehrere Ansätze.
Alle haben gemeinsam, dass Sender und Empfängerknoten verwendet werden. Dies
ist in zwei Richtungen möglich. Bei der ersten werden Sender im Raum verteilt
und der Empfänger bewegt sich. Die zweite Methode ist genau umgekehrt, hier
bewegt sich der Sender und wird von verteilten Empfängern wahrgenommen.

Die erste Methode ist die Laufzeitmessung der Signale. Da man die
Geschwindigkeit der Signale kennt, ist es möglich daraus die Distanz zwischen
Sender und Empfänger zu berechnen. Mit diesen Daten ist eine Triangulation
möglich.

Die zweite Methode ist die, die Ankunftsrichtung der Signale zu messen. Dazu werden
Empfänger benötigt, die eine Information darüber geben können aus welcher
Richtung Signale auftreffen.

Die beiden eben genannten Verfahren haben jedoch die Nachteile, das entweder
eine aufwendige Sensorhardware benötigt wird oder eine besondere Synchronisation
zwischen Sender und Empfänger voraus gesetzt werden muss. Um diesen Problemen
aus dem Weg zu gehen, haben Forscher der Princton University ein Verfahren
betrachtet, dass auf Basis der Signalstärke arbeitet. Bei diesem Versuch
wurden WLAN-Access-Points als Sender und Empfänger verwendet. Dabei wurden
die Entfernungen und ihre entsprechenden Signalstärken einmal empirisch
bestimmt und dann durch mathematische Methoden berechnet. Der Vorteil ist,
dass außer aufwendiger Berechnungen keine teure Hardware benötigt wurde.

%%%%%%%%%%%%%%%%%%%%%%%%%%%%%%%%%%%%%%%%%%%%%%%%%%%%%%%%%%%%%%%%%%%%%%%%%%%%
\cleardoublepage
