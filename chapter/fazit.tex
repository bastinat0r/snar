% vim:tw=78:ai:bg=light:set spell:spelllang=de:set nu
%%%%%%%%%%%%%%%%%%%%%%%%%%%%%%%%%%%%%%%%%%%%%%%%%%%%%%%%%%%%%%%%%%%%%%%%%%%%
%%% Fazit
%%%%%%%%%%%%%%%%%%%%%%%%%%%%%%%%%%%%%%%%%%%%%%%%%%%%%%%%%%%%%%%%%%%%%%%%%%%%
\chapter{Fazit}
Es gibt viele Möglichkeiten eine Indoor-Lokalisation auf die Beine zu
stellen. Je nach benötigter Präzision kann ein entsprechendes System
ausgewählt werden. Bei sehr geringer Präzision kann man sogar noch mit AGPS
arbeiten, wenn es nur darum geht einen Raum zu lokalisieren. Bei hoher
benötigter Präzision, wie beispielsweise beim Indoorflug, kann man auf
lasergestützes Tracking zurückgreifen.

Einen guten Mittelweg nimmt jedoch die ultraschallbasierte Lokalisation.
Sie ist genügend genau für die meisten Indoor-Anwendungen und bietet zu dem
noch zwei verschiedene Einsatzmöglichkeiten.
\dq Sprechende Infrastruktur\dq{} und \dq Sprechender Knoten\dq{} lassen
sich durchaus mit der selben Hardware realisieren.

%%%%%%%%%%%%%%%%%%%%%%%%%%%%%%%%%%%%%%%%%%%%%%%%%%%%%%%%%%%%%%%%%%%%%%%%%%%%
\cleardoublepage
