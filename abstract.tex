Positionsbestimmung in geschlossenen Räumen gewinnt immer mehr an Bedeutung.
Gerade auch in Bezug auf die Heimautomatisierung oder den Betrieb von
selbstständig arbeitenden Lagersystemen. Dabei gibt es viele Aspekte zu
beachten. Darunter fallen die Kosten, die Genauigkeit und Anwendbarkeit.
Dazu sollten verschiedene Systeme betrachtet werden, um abschätzen zu können,
welches System das jeweils passendste ist.

Diese Arbeit beschreibt einige Lokalisierungsmethoden für den Indoorbetrieb.
Dabei werden im ersten Teil vier wichtige Verfahren erklärt. Im zweiten Teil
der Arbeit wird das Ultraschallverfahren genauer betrachtet und auf zwei mögliche
Anwendungen eingegangen. Es werden weiterhin die Probleme und Vorteile dieses
Verfahrens behandelt.
